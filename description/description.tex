\documentclass[10pt,letterpaper]{article}
\usepackage[top=0.85in,left=2.75in,footskip=0.75in,marginparwidth=2in]{geometry}

% use Unicode characters - try changing the option if you run into troubles with special characters (e.g. umlauts)
\usepackage[utf8]{inputenc}

% clean citations
\usepackage{cite}

% hyperref makes references clicky. use \url{www.example.com} or \href{www.example.com}{description} to add a clicky url
\usepackage{nameref,hyperref}

% line numbers
\usepackage[right]{lineno}

% improves typesetting in LaTeX
\usepackage{microtype}
\DisableLigatures[f]{encoding = *, family = * }

% text layout - change as needed
\raggedright
\setlength{\parindent}{0.5cm}
\textwidth 5.25in 
\textheight 8.75in

% Remove % for double line spacing
%\usepackage{setspace} 
%\doublespacing

% use adjustwidth environment to exceed text width (see examples in text)
\usepackage{changepage}

% adjust caption style
\usepackage[aboveskip=1pt,labelfont=bf,labelsep=period,singlelinecheck=off]{caption}

% remove brackets from references
\makeatletter
\renewcommand{\@biblabel}[1]{\quad#1.}
\makeatother

% headrule, footrule and page numbers
\usepackage{lastpage,fancyhdr,graphicx}
\usepackage{epstopdf}
\pagestyle{myheadings}
\pagestyle{fancy}
\fancyhf{}
\rfoot{\thepage/\pageref{LastPage}}
\renewcommand{\footrule}{\hrule height 2pt \vspace{2mm}}
\fancyheadoffset[L]{2.25in}
\fancyfootoffset[L]{2.25in}

% use \textcolor{color}{text} for colored text (e.g. highlight to-do areas)
\usepackage{color}

% define custom colors (this one is for figure captions)
\definecolor{Gray}{gray}{.25}

% this is required to include graphics
\usepackage{graphicx}

% use if you want to put caption to the side of the figure - see example in text
\usepackage{sidecap}

% use for have text wrap around figures
\usepackage{wrapfig}
\usepackage[pscoord]{eso-pic}
\usepackage[fulladjust]{marginnote}
\reversemarginpar

% Adding multirow.
\usepackage{multirow}

% Other required things:
\usepackage{color}
\usepackage{subcaption}
\captionsetup[subfigure]{justification=centering}
\newcommand{\beachmat}{\textit{beachmat}}
\newcommand{\code}[1]{\texttt{#1}}

% document begins here
\begin{document}
\vspace*{0.35in}

% title goes here:
\begin{flushleft}
{\Large
    \textbf\newline{beachmat: a C++ API for data access from R matrix types}
}
\newline

% authors go here:
%\\
Aaron T. L. Lun\textsuperscript{1,*}
and friends
\\
\bigskip
\bf{1} Cancer Research UK Cambridge Institute, University of Cambridge, Li Ka Shing Centre, Robinson Way, Cambridge CB2 0RE, United Kingdom
\\
\bigskip
* aaron.lun@cruk.cam.ac.uk

\end{flushleft}

\section*{Abstract}
Blha blah blah

% now start line numbers
\linenumbers

\section*{Introduction}
Recent advances in single-cell RNA sequencing (scRNA-seq) technologies have led to an explosion in the quantity of data that can be generated in routine experiments.
Droplet-based methods such as Drop-Seq \cite{macosko2015highly}, inDrop \cite{klein2015droplet} and GemCode \cite{zheng2017massively} allow expression profiles to be captured for each of thousands to millions of cells.
It hardly needs to be said that this is a substantial amount of data -- the expression profile for each cell consists of a measure of expression for each transcriptionally active genomic feature, of which there are usually 10,000 to 40,000 in most common model organisms.
Careful computational analysis is critical to extract meaningful biology from these data, but the sheer volume strains existing pipelines and methods designed for single-cell data processing.
The challenge is compounded by the presence of large-scale projects such as the Human Cell Atlas \cite{regev2017human}, which aims to use single-cell `omics to profile every cell type in the human body.
Similar issues are encountered outside of transcriptomics, with single-cell ATAC-seq \cite{buenrostro2015single} and bisulfite sequencing \cite{smallwood2014single} providing region- to base-level resolution of biochemical events (chromatin accessibility and DNA methylation, respectively).
This results in even more data compared to gene-level expression values.

It is no exaggeration to say that the R programming language \cite{R} is the premier tool of choice for statistical data analysis.
R provides well-designed, rigorously-tested implementations of a large variety of statistical methods.
Its interactive nature makes it easy for newcomers to learn and lends itself to data exploration and research, while its programming features allow more experienced users to readily assemble complex analyses.
It is also extensible through the installation of optional packages, often contributed by the research community, which provide implementations of bespoke methods targeted to solve specific scientific problems.
In particular, the Bioconductor project \cite{gentleman2004bioconductor} supports a number of packages for biological data analyis, many of which focus on the processing of genomics data \cite{huber2015orchestrating}.
Packages are usually written in R but can also include compiled code (e.g., in C/C++ or Fortran), which is beneficial for computationally intensive tasks where high performance is required.
For C++ code, this process is facilitated by the \textit{Rcpp} package \cite{eddelbuettel2011seamless}, which simplifies the integration of package code with the R application programming interface (API).

In its simplest form, a scRNA-seq data set consists of a count matrix where each column is a cell, each row is a gene, and the value of each matrix entry is set to the quantified expression (e.g., number of mapped reads, transcripts-per-million) for that gene in that cell.
This can be most directly represented in R as a simple matrix, where each entry is explicitly stored in memory.
Alternatively, it can be stored as a sparse matrix using classes from the \textit{Matrix} package \cite{bates2017matrix}, which saves memory by only storing non-zero entries.
This exploits the fact that scRNA-seq protocols have low capture efficiencies \cite{grun2015design} -- RNA molecules are present in cells but are not reverse-transcribed to cDNA for sequencing, resulting in a preponderance of zeroes in the final count matrix.
Another option is to use file-backed representations such as those in the \textit{bigmemory} \cite{kane2013scalable} or \textit{HDF5Array} packages, where the data set is stored on disk and parts of it are extracted into memory upon request.
In each case, methods are provided in R for common operations such as subsetting, transposition and arithmetic, such that code written by users (or other developers) can be agnostic to the exact representation of the matrix.
This simplifies the development process and improves interoperability.

Unfortunately, for compiled code written in statically typed languages like C++, the details of the matrix representation must be known during compilation.
This makes it difficult to write a single, general piece of code that can be applied to many different representations.
Writing multiple versions for each representation is difficult and unsustainable when more representations become available.
The alternative is to perform all processing in R to exploit the availability of common methods.
However, this is an unappealing option for high-performance code.
For scRNA-seq data stored in matrices, consider the most common access pattern, i.e., looping across all cells or genes and performing operations on the cell- or gene-specific expression profiles.
If this was performed in R, the code within the loop would need to be re-interpreted at each of thousands or millions of iterations.
This increases the computational time required to perform analyses, which is inconvenient for small scripts, undesirable for interactive analyses and unacceptable for large simulation studies.
It would clearly be preferable to implement critical functions, loops and all, in compiled code wherever possible.

% One might think to simply move the inside of the loop in C++ to migitate the interpretation cost, while keeping the loop at the R level to access row- or column-level data.
% However, this involves some costs on its own, e.g., memory allocations that were previously one-offs need to be re-performed.
% If you need to share data across iterations, it will also involve more function calls to store relevant variables, so the cost of function calls is not avoided.

\section*{Description of the \beachmat{} API}

\subsection*{Overview}
The \beachmat{} API uses C++ classes to provide a common interface for data access from R matrix representations.
We define a base class that implements common methods for all matrix representations.
Each specific representation is associated with a derived C++ class that provides customized implementations of the access methods.
The intention is for a user to pass in an R matrix of any type, in the form of an \code{RObject} instance from the \textit{Rcpp} API (Figure X).
A function is then called to produce its C++ equivalent, returning a pointer to the base class.
This pointer is the same regardless of the R representation and can be used in downstream code to achieve run-time polymorphism.

While the API is agnostic to the matrix representation, it still needs to know the type of data that is stored within the matrix.
We use C++ templating to recycle the code to define specific classes for common data types, i.e., logical, integer, double-precision floating point or character strings.
The same methods are available for all classes of each data type, which eases the cognitive burden on the developer.
Briefly, when access to a specific row or column (or a slice thereof) is requested, the API will fill a \textit{Rcpp}-style \code{Vector} object with corresponding data values from the matrix.
A request for a specific entry of the matrix will directly return the corresponding data value.

In the following text, we discuss some of the specifics of the \beachmat{} API implementation.
This includes the details of each matrix representation, its memory footprint and the computational time required for data access.

\subsection*{Simple matrices}
By default, R stores matrices as one-dimensional arrays of length $N_rN_c$, where $N_r$ and $N_c$ are the number of rows and columns, respectively.
This is done in column-major format, i.e., the matrix entry $(x, y)$ corresponds to array element $x + N_ry$ (assuming zero-based indexing).
We refer to this format as a ``simple matrix''.
The simple matrix is easy to manipulate and the time required for data access is linear with respect to the number of rows/columns (Figure~\ref{fig:basetime}).
However, its memory footprint is directly proportional to its length.
For example, a double-precision matrix containing data for 10000 genes in each of one million cells would require 80 GB of RAM to store in memory.
This is currently not possible for most workstations, instead requiring dedicated high-performance computing resources.
Even smaller matrices will cause problems on systems with limited memory due to R's copy-on-write semantics.
Thus, the utility of simple matrices is limited to relatively small scRNA-seq data sets.

% While R doesn't copy if you don't modify, it _does_ copy if you do modify, rather tha overwriting the old version.
% Consider computing log2-normalized counts; it's four copies (first transpose, division, second transpose, log-transformation).

\begin{figure}[bt]
    \centering
    \begin{subfigure}[b]{0.49\textwidth}
        \includegraphics[width=\textwidth]{../timings/simulations/pics/base_col.pdf}
        \caption{}
    \end{subfigure}
    \begin{subfigure}[b]{0.49\textwidth}
        \includegraphics[width=\textwidth]{../timings/simulations/pics/base_row.pdf}
        \caption{}
    \end{subfigure}
    \caption{Time required for column or row access of simple matrices using \beachmat{} or \textit{Rcpp}.
        (a) Column access time with respect to increasing number of columns, for a matrix with 10000 rows.
        (b) Row access time with respect to increasing number of rows, for a matrix with 1000 columns.
        Access was timed after computing the column/row sums to ensure iteration across data values.
        Each time represents the average of 10 simulations (in milliseconds).
        Standard errors were negligible and are not shown.
    }
    \label{fig:basetime}
\end{figure}

We compare the access speed of the \beachmat{} API to that of a reference implementation using only \textit{Rcpp}.
Both row and column access via \beachmat{} require 20-50\% more time compared to the reference (Figure~\ref{fig:basetime}).
This is expected as \beachmat{} is built on top of \textit{Rcpp}, so the former cannot be faster than the latter.
Another reason is that, at each row/column access, \beachmat{} copies the matrix data into a \code{Vector}.
In contrast, the reference implementation avoids the overhead of creating a new copy by simply iterating across the original data.
Our use of copying is deliberate as it simplifies the API and keeps it consistent across matrix representations. 
Iterating across the original data is complicated for sparse matrices (due to lack of memory locality) and impractical for file-based representations.
Moreover, copying is required anyway for complex operations that involve transformations and/or re-ordering of data, as well as for libraries such as LAPACK that accept a pointer to a contiguous block of memory.
This suggests that, in practice, \beachmat{}'s computational overhead can be ignored.

\subsection*{Sparse matrices}
The \code{dgCMatrix} class from the \textit{Matrix} package stores sparse matrix data in compressed sparse column-orientated (CSC) format.
Consider that every non-zero entry in this matrix is characterized by a triplet: row $x$, column $y$ and value $v$.
To convert this into the CSC format, entries are sorted in order of increasing $x + N_ry$.
All entries with the same value of $y$ are now grouped together in the ordered sequence.
We refer to each column-based group as $G_y$, the entries of which are sorted internally in order of increasing $x$.
The representation is further compressed by discarding $y$ from each triplet.
All entries from the same column are at consecutive locations of the ordered sequence, so only the start position of $G_y$ on the sequence needs to be stored for each column.
(The end position of one column is simply the start position of the next column.)
This reduces memory usage to $s_IN_c + (s_I + s_v) N_{\ne 0}$ where $s_I$ is the size of an integer, $s_v$ is the size of a single data element and $N_{\ne 0}$ is the number of non-zero elements in the matrix.
For double-precision matrices with many rows, sparse matrices will be more memory-efficient than their simpler ``dense'' counterparts if the density of non-zero elements is less than $\approx66$\% (assuming 4-byte integers and 8-byte doubles). 

% Drops down to 33% for integer matrices, as Matrix needs to convert them to double-precision.

The CSC format simplifies data access by imposing structure on the non-zero entries.
When accessing a particular column $c$, all corresponding entries in $G_c$ can be quickly extracted by taking the relevant part of the ordered sequence.
For low-density sparse matrices, column access via \beachmat{} is even faster than access from simple matrices (Figure~\ref{fig:sparsecol}).
This is because only a few non-zero entries need to be copied -- the rest of the \code{Vector} can be rapidly filled with zeroes.
As the density of non-zero entries increases, column access becomes slower but is still comparable to that of simple matrices.
We note that the \textit{RcppArmadillo} package \cite{eddelbuettel2014arma} also handles sparse matrices via the \code{SpMat} class.
This provides faster column-level access than the \beachmat{} API as no copying of data is performed -- see above for a related discussion with simple matrices.

\begin{figure}[bt]
    \centering
    \begin{subfigure}[b]{0.49\textwidth}
        \includegraphics[width=\textwidth]{../timings/simulations/pics/sparse_col_density.pdf}
        \caption{}
    \end{subfigure}
    \begin{subfigure}[b]{0.49\textwidth}
        \includegraphics[width=\textwidth]{../timings/simulations/pics/sparse_col_ncol.pdf}
        \caption{}
    \end{subfigure}
    \caption{Column access times for CSC matrices using \beachmat{} (sparse) or \textit{RcppArmadillo} (arma), compared to access times for an equivalent simple matrix with \beachmat{} (dense).
        (a) Access times with respect to increasing density of non-zero entries (as a percentage of all entries), for a matrix with 10000 rows and 1000 columns.
        (b) Access times with respect to increasing number of columns, for a matrix with 10000 rows and 1\% non-zero entries.
        Timings were computed after taking the column sums.
        Each time represents the average of 10 simulations (with negligible standard errors).
    }
    \label{fig:sparsecol}
\end{figure}

Row-level access is more difficult in the CSC format as entries in the same row do not follow a predictable pattern.
If a row is requested, a binary search on $x$ needs to be performed within $G_y$ for each column $y$, which requires an average time proportional to $\log(N_r)$.
In contrast, obtaining the next element in a row of a dense matrix can be done in constant time by jumping $N_r$ elements ahead on the one-dimensional array.
To speed up row access for sparse matrices, we realized that the most common access pattern involves requests for consecutive rows.
If row $r$ is accessed, \beachmat{} will loop over all columns and cache the index the first value of $x$ in $G_y$ that is not less than $r$.
When $r+1$ is accessed, we simply need to check if each of the indices should be incremented by one.
This avoids the need to perform a new binary search and reduces the row access time substantially (Figure~\ref{fig:sparserowrand}a, b).
Even when the row access pattern is random, we mitigate the time penalty by checking if the requested row is greater than or less than the previous row for which indices are stored.
If greater, we use the stored indices to set the start of the binary search; if less, we use the indices to set the end of the search. 
This reduces the search space and the amount of computational work for large $N_r$.

\begin{figure}[bt]
    \begin{subfigure}[b]{0.49\textwidth}
        \includegraphics[width=\textwidth]{../timings/simulations/pics/sparse_row_rand_density.pdf}
        \caption{}
    \end{subfigure}
    \begin{subfigure}[b]{0.49\textwidth}
        \includegraphics[width=\textwidth]{../timings/simulations/pics/sparse_row_rand_nrow.pdf}
        \caption{}
    \end{subfigure}
    \begin{subfigure}[b]{0.49\textwidth}
        \includegraphics[width=\textwidth]{../timings/simulations/pics/sparse_row_density.pdf}
        \caption{}
    \end{subfigure}
    \begin{subfigure}[b]{0.49\textwidth}
        \includegraphics[width=\textwidth]{../timings/simulations/pics/sparse_row_nrow.pdf}
        \caption{}
    \end{subfigure}
    \caption{Row access times for CSC matrices using \beachmat{}, using consecutive or randomly selected rows (a, b) or in comparison to an equivalent simple matrix with \beachmat{} (c, d).
        (a,c) Access times with respect to increasing density of non-zero entries, for a matrix with 10000 rows and 1000 columns.
        (b, d) Access times with respect to increasing number of rows, for a matrix with 1000 columns and 1\% non-zero entries.
        Timings were computed after taking the row sums.
        Note that consecutive row access was used in (c) and (d) for both sparse and simple (dense) matrices.
    }
    \label{fig:sparserow}
\end{figure}


Despite these optimisations, consecutive row access with sparse matrices in \beachmat{} remains slower than that with simple matrices (Figure~\ref{fig:sparserow}c, d).
This is not surprising as there is simply less work to do with dense matrices.
The exception is with large matrices of low density, where the cost of cache misses due to large jumps exceeds that of the binary search.
However, both are substantially faster than row access with \textit{RcppArmadillo}.
For example, for a 10000-by-1000 sparse matrix with 1\% non-zero entries, \beachmat{} with sparse matrices takes 39.8 milliseconds to access each row; \beachmat{} with dense matrices takes 29.2 milliseconds; and \textit{RcppArmadillo} takes 1921.4 milliseconds.
These results motivate the use of \beachmat{} for data access.

{\small
    \bibliography{ref}
    \bibliographystyle{abbrv}
}


\end{document}
